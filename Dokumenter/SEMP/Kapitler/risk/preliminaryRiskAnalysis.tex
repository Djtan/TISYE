\section{Preliminary Risk Analysis}
In order to prevent cases of issues a preliminary risk analysis has been draft. It identifies future possible problems. These are then assigned values for likelihood of occurring and what the consequences will be if not solved. 

\begin{table}[H]
\begin{tabular}{|l|l|l|l|}
\hline
\textbf{\#} & \textbf{Identified Risks} & \textbf{L} & \textbf{C}\\ \hline
1 & Too expensive hardware platform for dismounted COP. & 6 & 6. 
\\  \hline

2 &Realtime communication is too slow. & 3 & 8. 
\\  \hline

3 &Device tracking is not accurate enough. & 4 & 6. 
\\  \hline

4 &Safety sensors is not accurate enough. & 3 & 8. 
\\  \hline

5 &Environment requirements of the dismounted COP cannot be met.  & 3 & 6. 
\\  \hline

6 &Size and design of dismounted COP makes it unwearable. & 2 & 9. 
\\  \hline

7 &Database is full. & 1 & 10.
\\ \hline

8 &Hardware breaks. & 8 & 10.
\\ \hline

\label{table_risks}
\end{tabular}
\caption{Identified Risks (L:likelihood, C: consequence).}
\end{table}


\begin{center}
\begin{figure}[H]
\centering
\includegraphics[width=0.75\textwidth]
{Billeder/risk.png}
\caption{Risk Analysis}
\label{fig:risk_analysis}
\end{figure}
\end{center}

On the graf in figure \ref{fig:risk_analysis} all identified problems have been depicted. This has been done in order to illustrate how great the risks involved are. The risks identified need special attention during the development phase. 