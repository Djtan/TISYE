\section{Deliverables and outcome}
In this section the wanted format of a response to the RFP is elaborated. All documents to be exchanged shall be in portable document format.

\subsubsection{Cover letter}
A covering letter, dated and signed by a person authorized to negotiate, make commitments, and provide any clarifications with respect to the proposal on behalf of the bidding consultant or firm. Provide a statement indicating your company’s understanding of the proposed project and the deliverables required. This is done to match expectation of the project. Optionally provide an indication of any proposed deviations or exceptions to the terms and conditions outlined in this RFP document.

\subsubsection{Scope}
Discuss in detail items from the RFP and how you intend to tackle it. Use diagrams to illustrate your configuration. This will be the longest section of your proposal and will probably have several subsections

\subsubsection{Proposed project schedule and cost}
When do you anticipate starting? How long will each task take? Make a table of your expected schedule for completing the project.

Breakdown the cost by equipment and personnel time to come up with your expected budget. The budget must include all phases and describe in details. 

\subsubsection{Supporting Information}
Any other information regarding the project. This could be technology considerations. It could also be what your company have done in similar projects. 